\documentclass[12pt]{article}
 
\usepackage[margin=0.8in]{geometry}
\usepackage{amsmath}
\usepackage{amssymb}
\usepackage{graphicx}
\usepackage{tikz}
\usepackage{pgfplots}

\pgfplotsset{compat=1.18}

\title{Precalculus: Functions and Their Graphs - Homework Solutions}
\author{Aiden Ryu}
\date{\today}

\begin{document}

\maketitle

\section*{Solutions}

\begin{enumerate}

\item The slope is $m = \frac{8-4}{3-1} = \frac{4}{2} = 2$. Using point-slope form with (1, 4):
   $y - 4 = 2(x - 1)$
   $y = 2x - 2 + 4$
   $y = 2x + 2$

\item Given $f(x) = x^2 - 2x + 3$ and $g(x) = x + 1$:
\begin{itemize}
    \item[(a)] $(f \circ g)(x) = f(g(x)) = (x+1)^2 - 2(x+1) + 3 = x^2 + 2x + 1 - 2x - 2 + 3 = x^2 + 2$
    \item[(b)] $(g \circ f)(x) = g(f(x)) = (x^2 - 2x + 3) + 1 = x^2 - 2x + 4$
    \item[(c)] $(f \circ g)(2) = 2^2 + 2 = 6$
\end{itemize}

\item Starting with $f(x) = |x|$, to get $g(x) = |x - 1| + 2$:
   1. Shift 1 unit right: $|x - 1|$
   2. Shift 2 units up: $|x - 1| + 2$

\item Let $f(x) = x^2 + 1$ and $g(x) = 2x - 3$.
\begin{itemize}
    \item[(a)] $(f + g)(x) = f(x) + g(x) = (x^2 + 1) + (2x - 3) = x^2 + 2x - 2$
    \item[(b)] $(f \cdot g)(x) = f(x) \cdot g(x) = (x^2 + 1)(2x - 3) = 2x^3 - 3x^2 + 2x - 3$
    \item[(c)] $(f + g)(3) = 3^2 + 2(3) - 2 = 9 + 6 - 2 = 13$
               $(f \cdot g)(3) = (3^2 + 1)(2(3) - 3) = 10 \cdot 3 = 30$
\end{itemize}

\item For $f(x) = 3x + 4$:
   $y = 3x + 4$
   $x = \frac{y-4}{3}$
   $f^{-1}(x) = \frac{x-4}{3}$
   
   $f^{-1}(10) = \frac{10-4}{3} = \frac{6}{3} = 2$

\item For $h(x) = x^2 - 4x - 5$:
\begin{itemize}
    \item[(a)] y-intercept: $h(0) = 0^2 - 4(0) - 5 = -5$, so (0, -5)
    \item[(b)] x-intercepts: $x^2 - 4x - 5 = 0$
               $(x-5)(x+1) = 0$
               $x = 5$ or $x = -1$
    \item[(c)] Vertex: $x = -b/(2a) = -(-4)/(2(1)) = 2$
               $y = 2^2 - 4(2) - 5 = 4 - 8 - 5 = -9$
               Vertex is (2, -9)
    \item[(d)] The parabola opens upward because the coefficient of $x^2$ is positive.
\end{itemize}

\item For $f(x) = \frac{x + 2}{x - 1}$:
\begin{itemize}
    \item[(a)] Domain: All real numbers except 1 (where denominator equals zero)
    \item[(b)] Vertical asymptote: $x = 1$
               Horizontal asymptote: As $x \to \infty$, $y \to 1$
    \item[(c)] [Graph would be sketched here]
\end{itemize}

\item $f(x) = x^3 - x$ is one-to-one. Proof: If $f(a) = f(b)$, then $a^3 - a = b^3 - b$.
   $a^3 - b^3 = a - b$
   $(a-b)(a^2+ab+b^2) = a - b$
   $(a-b)(a^2+ab+b^2-1) = 0$
   Either $a-b = 0$ or $a^2+ab+b^2-1 = 0$. The second equation has no real solutions for $a \neq b$.
   Therefore, $a = b$, proving the function is one-to-one.

\item For $f(x) = \sqrt{x + 2}$:
   Domain: $x + 2 \geq 0$, so $x \geq -2$
   Range: $y \geq 0$, so $[0, \infty)$

\item For $f(x) = 2^x$, to get $g(x) = 2^{x+1} - 1$:
   1. Shift 1 unit left: $2^{x+1}$
   2. Stretch vertically by a factor of 2: $2 \cdot 2^x$
   3. Shift 1 unit down: $2^{x+1} - 1$

\item $\log_2(x + 3) = 4$
   $2^4 = x + 3$
   $16 = x + 3$
   $x = 13$

\item $x^2 - 2x - 3 = 2x + 1$
   $x^2 - 4x - 4 = 0$
   $(x-4)(x+0) = 0$
   $x = 4$ or $x = 0$

\item Given $f(x) = 3x - 1$ and $g(x) = \frac{x}{2} + 1$:
   $(f \circ g)(x) = f(g(x)) = 3(\frac{x}{2} + 1) - 1 = \frac{3x}{2} + 3 - 1 = \frac{3x}{2} + 2$
   $(g \circ f)(x) = g(f(x)) = \frac{3x-1}{2} + 1 = \frac{3x-1+2}{2} = \frac{3x+1}{2}$

\item $y = |x^2 - 4|$ has:
   - y-intercept at (0, 4)
   - x-intercepts at (-2, 0) and (2, 0)
   - Vertex at (0, 4)
   - Minimum points at (-2, 0) and (2, 0)

\item The graph of $f(x) = \frac{1}{x-2}$ is the graph of $y = \frac{1}{x}$ shifted 2 units to the right.

\end{enumerate}

\end{document}